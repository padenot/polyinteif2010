Dans cette partie, on va essayer de vous présenter les lieux phares de votre scolarité : 

\paragraph{Le département}
Situé à proximité du Grand Restaurant, et de l'arrêt de tram' Gaston Berger
(Ligne T1), le département informatique se situe en réalité aux deuxième et
troisième étages du bâtiment Blaise Pascal, le premier étant occupé par le
département Sciences et Génie des Matériaux (vous les reconnaîtrez facilement :
ils sont toujours devant vous à la machine à café ... ahem).

Au deuxième étage, vous trouverez les salles de TP (5 salles Windows dont trois
avec des machines déportés, et une salle Linux) ainsi qu'une salle de
TD/TP où vous effectuerez les phases de conception. Il est à noter que deux
salles Java Sun Spot sont aussi mises à disposition à cet étage. Enfin tout
cela vous sera rappelé lors d'un TP de présentation en début d'année !

Au troisième étage se trouvent les salles de TD, rien d'exceptionnel, ce sont
des salles avec des tables, des chaises, un tableau, et un ordinateur pour le
prof (car on est en informatique quand même). 

Chose importante, nous partageons aussi nos locaux avec plusieurs laboratoires
de recherche en informatique, nous sommes donc quasiment tout le temps en
contact avec les chercheurs du LIRIS\footnote{\url{http://liris.cnrs.fr/}} entre autres, qui sont aussi pour certains
nos enseignants, et ça, c'est super pour ceux qui sont intéressés par ce domaine !

\paragraph{L'Amphi Gaston Berger}
Lieu culte, voire mythique, certaines légendes estudiantines racontent que l'on
y dispense des savoirs en informatique...  Malheureusement, le chemin reste
encore ... 

Ah on me signale à l'oreillette qu'il s'agit en réalité de l'amphi où se
déroulent tous les cours magistraux qui vous seront donnés en 3 et 4 IF...vous
allez donc y passer un certain nombre d'heures ! 

\paragraph{Le bâtiment des humanités}
Situé à mi-chemin entre le département et la Maison des étudiants, c'est ici
que vous aurez tous vos cours de tout ce qui n'est pas informatique... C'est à
dire Sciences Humaines et communications (SHC) et les langues vivantes. 
Ce bâtiment est le centre névralgique de l'activité culturelle à l'Insa, et il
est souvent le cadre d'expositions d'art sympathiques pour peu que l'on apprécie !

\paragraph{Le Grand Restaurant}
Autre lieu important du campus, le Grand Restaurant principal de l'Insa, où vous
serez conviés à vous repaître tout au long de l'année ! 

La chose fourbe de l'histoire, c'est qu'il abrite en réalité un grand
restaurant, mais aussi un snack plus connu sous le nom de Prévert qui s'avère
relativement pratique pour les soirs où on a pas le temps de manger, mais ce
qui est servi est pour la plupart du temps «~un peu lourd à digérer~».  

À noter que deux restaurants universitaires sont présents juste à l'extérieur du campus et proposent des repas à l'unité. Utile pour les potentiels externes !

\paragraph{La Maison des étudiants}
Lieu phare de l'INSA, ce bâtiment situé juste derrière celui des humanités
abrite le coeur de l'associatif Insalien, c'est à dire le Bureau des Élèves (ou
BdE), une rapide présentation vous est faîte un peu plus loin dans le
poly. Pour continuer dans le domaine associatif, le bâtiment abrite aussi le
club lecture (où vous trouverez une foison de bouquins/BD/Mangas), un salon de
lecture/réunion et très prochainement une salle télé!
Ce bâtiment abrite aussi les deux autres restaurants de l'Insa :
\subparagraph{Le Grillon}
Restaurant qui illustre parfaitement la citation suivante «~\emph{Plus c'est long,
	   plus c'est bon!}~».

N'y voyez aucun sous entendu, c'est juste que ce qui est servi là-bas est probablement
ce que vous pourrez avoir de meilleur sur le campus. En revanche ça se paye car le restaurant 
est généralement surchargé et il faut donc se munir de patience ! Avis aux Gastronomes ! 
\subparagraph{L'Olivier}
Restaurant  spécialisé dans les pizza/pâtes et autres plats venus du fin
fond de l'Italie, on y accède un peu plus rapidement qu'au Grillon et on y
mange le plus souvent pas mal ! Attention tout de même, ça reste des
pizzas ! 

\paragraph{La K-Fêt}
Envie d'aller vous jeter un godet après une rude journée ? Envie d'aller
décompresser à la fin des partiels, ne cherchez plus : La K-fêt est faîte
pour vous !

Lieu mythique des soirées Insaliennes, la K-fêt se trouve dans le patio de
la maison des élèves. C'est le Bar associatif de l'Insa, tenu par des
étudiants volontaires !

Elle vous assure boissons et bonne ambiance quasiment tout les soirs, et à
moindre coût !

Elle organise le jeudi précédent chaque vacances une «~Boom K-Fêt~» pour fêter
le début des vacances, soirées mémorables pour la plus grande majorité ! À ne
pas rater !

À noter que l'équipe de l'Inté'IF y sera en soirée pendant les 2 ou 3 jours qui
précèdent le début de vos cours, donc si vous voulez passer nous rencontrer et
accessoirement passer une bonne soirée, n'hésitez plus ! On sera de toute façon
facilement reconnaissables, et nos coordonées vous sont données dans le
paragraphe Planning (page \pageref{rplanning}) de l'inté !

\paragraph{La Rotonde}
Et oui, sous ce nom se cache en réalité la salle de spectacles de l'Insa !
Salle entièrement gérée par une association d'étudiants, elle a une
capacité de 500 places. Elle est à la disposition des autres associations, avec
la prestation technique (lumières, décors...).

Ansi, toute l'année est ponctuée par différentes animations :
\begin{itemize}
\item  Séance Cinéma : CinéClub
\item  Pièces de théâtre : TTI et section Théâtre-études
\item  Spectacle de Danse : Section Danse-études
\item  Concerts Ziket : Section Musique Etudes / Association Musicale de l'INSA (AMI) 
\end{itemize}

\paragraph{Les logements}
Si tu es admis direct, sache que tu as toutes les chances d'avoir un logement
dans une des nombreuses résidences de l'INSA. La qualité est plus que correct,
mais quelques points jouent en défaveur du choix de cette solution : la qualité
de la connexion internet, qui permet de travailler, tout au plus (quand il y a
vent arrière), et l'obligation de manger matin, midi et soir aux
restaurants de l'INSA, ce qui pourrait en lasser certains, à la longue.
Enfin, ne nous voilons pas la face, c'est quand même une excellente expérience,
et être a 1 minutes 30 du bâtiment IF à pieds a certains avantages
indéniables.

Ces magnifiques oeuvres de l'architectures insalienne se composent, pour la
grande majoritée, d'un sommier, d'un matelas (les draps, couvertures,
oreillers, ne sont pas fournis), de plaques électriques, d'un frigo,
d'une douche, WC, d'un bureau (plus ou moins grand selon la taille de la
chambre), de nombreuses prises de courant, d'une prise ethernet (la
connexion se fait par VPN Cisco) et TV (câbles non fournis, mais
achetables au BdE si besoin), des étagères sont présentes, ou
non selon les chambres (les rangements en plastiques Made in Leroy Merlin sont
un bon remplacement), une chaise (de facture douteuse).

Du côté de l'accueil, et bien à votre arrivée dans la résidence, un responsable
ou un étudiant étant là pour aider ce dernier, vous emmènera visiter votre
nouvel antre en échange du bon d'emménagement que vous aurez préalablement
récupéré à la Direction Des Résidences, après une visite expédiée aussi vite
qu'une lettre à la poste, vous voilà chez vous (n'oubliez pas de regarder
l'état des chauffages, lumières, volets et fenêtres, lors de la visite
avec le régisseur, sait on jamais...)

Vous voilà chez vous, à deux portes d'autres IFs.
