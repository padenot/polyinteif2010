Dans cette partie, on va essayer de vous présenter les lieux phares de votre scolarité : 

\paragraph{Le département}
Situé à proximité du Grand Restaurant, et de l'arrêt de tram Gaston Berger
(Ligne T1), le département informatique se situe en réalité aux deuxième et
troisième étages du bâtiment Blaise Pascal, le premier étant occupé par le
département Sciences et Génie des Matériaux (vous les reconnaîtrez facilement :
ils sont toujours devant vous à la machine à café... ahem).

\vspace{1em}

Au deuxième étage, vous trouverez les salles de TP (5 salles Windows dont trois
avec des machines déportés, et une salle Linux) ainsi qu'une salle de
TD/TP où vous effectuerez les phases de conception des projets. Il est à noter que deux
salles Java Sun Spot sont aussi mises à disposition à cet étage. Tout
cela vous sera rappelé lors d'un TP de présentation en début d'année.

\vspace{1em}

Au troisième étage se trouvent les salles de TD : rien d'exceptionnel, ce sont
des salles avec des tables, des chaises, un tableau, et un ordinateur pour le
prof (car on est en informatique tout de même !).

\vspace{1em}

Chose importante, nous partageons aussi nos locaux avec plusieurs laboratoires
de recherche en informatique, nous sommes donc quasiment tout le temps en
contact avec les chercheurs du LIRIS\footnote{\url{http://liris.cnrs.fr/}}, entre autres, qui sont aussi pour certains
nos enseignants, et ça, c'est super pour ceux qui sont intéressés par ce domaine !

\paragraph{L'Amphi Gaston Berger}
Lieu culte, voire mythique, certaines légendes estudiantines racontent que l'on
y dispense des savoirs mystérieux en informatique... Malheureusement, le chemin reste
encore long pour... Ah, on me signale à l'oreillette qu'il s'agit en réalité de l'amphi où se
déroulent tous les cours magistraux qui vous seront donnés en 3 et 4 IF... Vous
allez donc y passer un certain nombre d'heures, alors courage ! 

\paragraph{Le bâtiment des Humanités}
Situé à mi-chemin entre le département et la Maison des Étudiants, c'est ici
que vous aurez tous vos cours de tout ce qui n'est pas informatique... C'est à
dire Sciences Humaines et Communication (SHC) et les langues vivantes. 
Ce bâtiment est le centre névralgique de l'activité culturelle à l'INSA, et il
est souvent le cadre d'expositions d'art sympathiques pour peu que l'on apprécie ce sujet !

\paragraph{Le Grand Restaurant}
Autre lieu important du campus, le Grand Restaurant de l'INSA, où vous
serez conviés à vous repaître tout au long de l'année ! La chose fourbe de l'histoire, c'est qu'il abrite en réalité un restaurant, surnommé le «~Beurk~» (vous verrez vite pourquoi), mais aussi
un snack, le Prévert, plus connu sous le nom de Pervert, qui s'avère
relativement pratique les midis/soirs où l'on a pas beaucoup de temps pour manger, mais ce
qui est servi est la plupart du temps «~un peu lourd à digérer~».

\vspace{1em}

À noter que deux autres restaurants universitaires sont présents juste à l'extérieur du campus et proposent des repas à l'unité. Utile pour les potentiels externes !

\paragraph{La Maison des Étudiants}
Lieu phare de l'INSA, ce bâtiment situé juste derrière celui des Humanités
abrite le coeur de l'associatif Insalien, c'est à dire le Bureau des Étudiants (ou
BdE), dont une rapide présentation vous est faîte un peu plus loin dans le
poly. Pour continuer dans le domaine associatif, le bâtiment contient le
club lecture (où vous trouverez une foison de bouquins/BD/mangas), un salon de
lecture/réunion et très prochainement une salle télé !
Ce bâtiment abrite aussi les deux autres restaurants de l'INSA :

\subparagraph{Le Grillon}
Restaurant qui illustre parfaitement la citation suivante «~\emph{Plus c'est long,
	   plus c'est bon !}~». N'y voyez aucun sous-entendu, c'est juste que ce qui est servi là-bas est probablement
ce que vous pourrez avoir de meilleur sur le campus. En revanche, ça se paye car le restaurant 
est généralement surchargé et il faut donc se munir de patience pour pouvoir y manger ! Avis aux gastronomes !

 \vspace{1em}
 
\subparagraph{L'Olivier}
Restaurant  spécialisé dans les «~pizza/pâtes~» et autres plats venus du fin
fond de l'Italie, on y accède un peu plus rapidement qu'au Grillon et on y
mange assez bien le plus souvent !

\paragraph{La K-Fêt}
Envie d'aller vous jeter un godet après une rude journée ? Envie d'aller
décompresser à la fin des partiels ? Ne cherchez plus : La K-Fêt est faite
pour vous !

\vspace{1em}

Lieu mythique des soirées Insaliennes, la K-Fêt se trouve dans le patio de
la maison des élèves. C'est le bar associatif de l'INSA, tenu par de courageux
étudiants volontaires !

\vspace{1em}

Elle vous assure boissons et bonne ambiance quasiment tous les soirs, et à
moindre coût !

\vspace{1em}

Le jeudi précédent chaque période de vacances, une «~Boom K-Fêt~» est organisée
pour fêter la fin des cours, soirées mémorables pour la majorité ! À ne
pas rater !

\vspace{1em}

À noter que l'équipe d'inté IF y sera en soirée pendant les 2 ou 3 jours qui
précèdent le début de vos cours, donc si vous voulez nous rencontrer et
accessoirement passer une bonne soirée, n'hésitez plus ! On sera de toute façon
facilement reconnaissables, et nos coordonnées vous sont données dans le
paragraphe Planning (page \pageref{rplanning}) du poly !

\paragraph{La Rotonde}
Et oui, sous ce nom se cache en réalité la salle de spectacles de l'INSA !
Entièrement gérée par une association d'étudiants, elle a une
capacité de 500 places, et se met à la disposition des autres associations, avec
des prestations techniques (lumières, décors...) dignes de ce nom !

\vspace{1em}

Ansi, toute l'année est ponctuée par différentes animations :
\begin{itemize}
\item  Séance cinéma : CinéClub
\item  Pièces de théâtre et matchs d'impro : TTI et section Théâtre-études
\item  Spectacle de Danse : section Danse-études
\item  Concerts Ziket : Section Musique-études / Association Musicale de l'INSA (AMI) 
\end{itemize}

\paragraph{Les logements}
Si tu es admis direct, sache que tu as toutes tes chances d'avoir un logement
dans l'une des nombreuses résidences de l'INSA. La qualité est plus que correcte,
mais quelques points peuvent jouer en défaveur de cette solution : la connexion internet, qui
permet de travailler, tout au plus (quand il y a
vent arrière), et l'obligation de manger matin, midi et soir aux
restaurants de l'INSA, ce qui pourrait en lasser certains à la longue (rassures-toi, jusqu'à présent personne n'en est mort).
Enfin, ne nous voilons pas la face, c'est quand même une excellente expérience,
et habiter à 1 minute 30 à pied du bâtiment IF et des restaurants du campus a certains avantages indéniables.

\vspace{1em}

Ces magnifiques oeuvres de l'architecture Insalienne se composent, pour la
grande majoritée, d'un sommier, d'un matelas (les draps, couvertures et
oreillers ne sont pas fournis), d'une ou deux plaques électriques, d'un frigo,
d'une douche, de toilettes, d'un bureau (plus ou moins grand selon la taille de la
chambre), de nombreuses prises de courant, d'une prise Ethernet (la
connexion se fait par VPN) et TV (câbles non fournis, mais
achetables au BdE si besoin), sans oublier les étagères (plus ou moins grandes
selon les chambres : les rangements en plastiques sont
un bon remplacement) et une chaise (de facture douteuse : mieux vaut prévoir de
s'acheter un bon gros fauteuil de bureau bien confortable !). C'est à peu près tout, et c'est déjà pas mal !

\vspace{1em}

Du côté de l'accueil, à ton arrivée dans la résidence, un responsable t'emmènera visiter ton
nouvel antre en échange du bon d'emménagement que tu auras préalablement
récupéré à la Direction Des Résidences. Après une visite expédiée aussi vite
qu'une lettre à la poste, tu es désormais chez toi (n'oublie pas de regarder
l'état des chauffages, lumières, volets, fenêtres,  etc. lors de la visite
avec le régisseur, on sait jamais...) !

\vspace{1em}

Te voilà maintenant dans ta nouvelle turne, à deux portes d'autres IFs, en entouré par de gentils étudiants !
N'hésite pas à faire connaissance avec le voisinage, on ne sait jamais quelle surprises il réserve !
