\begin{citationii}
\emph{L'INSA c'est bien beau, mais au final on y fait quoi ?}
\end{citationii}
Au final vous vous êtes engagés à l'Insa, mais vous vous posez tous cette
question !
Pas de panique, vos super parrains sont là pour y répondre!
La scolarité à l'INSA est basée sur un model d'université, c'est à dire cours
magistraux en amphi, Travaux Dirigés (TD) afin de garantir la compréhension du
cours et enfin les Travaux Pratiques (TP) pour appliquer.

\subsubsection{Amphis}
Les cours à l'Insa se déroulent sous forme de cours magistraux d'une durée de
1h30, pendant les matinées, ce qui vous
laisse donc deux cours par matinées. Les Cours sont organisés en modules qui
seront traités de manière cohérente par la suite dans les TD / TP.

La présence en cours est obligatoire, mais non vérifiée. Le meilleur conseil que
nous puissions donner vis à vis de cela est que «~ce qui est fait n'est plus à
faire, et ce qui n'est pas fait reste toujours à faire~». En d'autres termes, les
amphis peuvent vous permettre d'assimiler le bagage théorique rapidement, et bien
entendu, il existe de nombreux moyens d'acquérir ce bagage, mais l'amphi, à
condition qu'il soit bien suivi et noté, reste le moyen de loin le plus efficace
! Après vous savez probablement bien mieux que nous ce dont vous avez besoin!
\subsubsection{TD}
Les TD vous occuperons deux après-midis sur les quatre de cours dans la semaine.
L'objectif ici est de mettre en pratique les cours de façon à mieux les
maîtriser, c'est aussi l'occasion d'avoir un échange privilégié avec
l'enseignant pour pouvoir poser des questions. Les TD à l'INSA n'ont pas grand
chose de différent de ceux que vous avez déjà eu l'occasion d'avoir ! 

Truc sympa, c'est que les profs pour la grande majorité ne voient aucun
inconvénient à ce que vous fassiez de la prise de note sur ordinateur, pour
certaines matières c'est assez infaisable (nos meilleurs experts en \LaTeX{} se
sont brisés le nez en maths...), mais ça reste pratique... Enfin les
amateurs du papier-crayons restent majoritaires.
\subsubsection{TP}
Les TPs sont l'élément clé de l'enseignement en troisième année. En effet, ils
vous donneront l'occasion d'appliquer le bagage de connaissances acquis dans
les cours et approfondis dans les TD. Deux séances de TP (de 4h chacune) sont
organisées par semaine.

Ils peuvent être de formes multiples (TP papier, programmation diverse, etc.)
    et de durée variable. Par exemple certains sont à rendre en fin de séance
    d'autres ont un délai de 5 semaines maximum à partir de la première séance.
    

Les TP sont aussi l'occasion d'apprendre le travail en équipe. En effet pour la
3IF vous vous verrez assigné à trois autres personnes (une par trimestre) qui
seront vos binômes, et avec qui il faudra se répartir le travail.

Vous l'aurez compris, une des plus grosses difficultés de la 3IF est de réussir
à gérer correctement l'ordonnancement des TP de façon à pouvoir tous les
traiter manière suffisante, alors soyez prévoyants  si vous ne voulez pas
exploser votre budget caféine !  

(Avec du recul, c'est comme de l'ordonnancement de tâches sur un processeur
 double coeur...) 

(Attention à la sur-qualité : c'est de la non-qualité, c'est un principe de
 base en informatique, et on vous le répètera bien assez souvent !).
