\vspace{1em}
\begin{citationii}
\emph{L'INSA c'est bien beau, mais au final on y fait quoi ?}
\end{citationii}
Au final, vous vous êtes engagés à l'INSA, mais vous vous posez tous cette
question !
Pas de panique, vos supers parrains sont là pour y répondre !
La scolarité à l'INSA est basée sur un modèle d'université, c'est-à-dire cours
magistraux en amphi, Travaux Dirigés (TD) afin de garantir la compréhension des
cours et enfin Travaux Pratiques (TP) pour les appliquer.

\subsubsection{Amphis}
Les cours en IF se déroulent sous forme de cours magistraux d'une durée de
1h30, à raison de deux par matinée. Ils sont organisés en modules qui
seront traités de manière cohérente par la suite dans les TD/TP.

\vspace{1em}

La présence en cours est obligatoire, mais non vérifiée. Le meilleur conseil que
nous puissions donner vis-à-vis de cela est que «~ce qui est fait n'est plus à
faire, et ce qui n'est pas fait reste toujours à faire~». En d'autres termes, les
amphis peuvent vous permettre d'assimiler le bagage théorique assez rapidement, et bien
entendu, il existe de nombreux moyens d'acquérir ce bagage, mais l'amphi, à
condition qu'il soit bien suivi et accompagné de prises de notes, reste le moyen le plus 
efficace (et de loin) ! Après vous savez probablement bien mieux que nous ce dont vous avez besoin !
\subsubsection{TD}
Les TD vous occuperons deux après-midis sur les quatre de cours dans la semaine.
L'objectif ici est de mettre en pratique les cours de façon à mieux les
maîtriser, et c'est aussi l'occasion d'avoir un échange privilégié avec
l'enseignant pour pouvoir poser des questions. Les TD à l'INSA n'ont pas grand chose
de différent avec ceux que vous avez déjà eu l'occasion d'avoir ! 

\vspace{1em}

Truc sympa, c'est que les profs, pour la grande majorité, ne voient aucun
inconvénient à ce que vous fassiez de la prise de note sur ordinateur. Pour
certaines matières c'est assez infaisable (nos meilleurs experts en \LaTeX{} se
sont brisés le nez en maths...), mais ça reste pratique. Toutefois les
amateurs du papier-crayon restent majoritaires.
\subsubsection{TP}
Les TP sont l'élément clé de l'enseignement en 3IF. En effet, ils
vous donneront l'occasion d'appliquer le bagage de connaissances acquis dans
les cours et approfondi dans les TD. Deux séances de TP (de 4h chacune) sont
organisées par semaine.

\vspace{1em}

Ils peuvent être de formes multiples (TP papier, programmation, etc.)
    et de durée variable. Par exemple, certains sont à rendre en fin de séance,
    d'autres ont un délai de 5 semaines maximum à partir de la première séance (attention
	avec ces derniers, le délai accordé est souvent piégeur, et il ne faut pas s'y prendre à la dernière minute).
    
\vspace{1em}

Les TP sont aussi l'occasion d'apprendre le travail en équipe. En effet, pour toute la
3IF, vous travaillerez en collaboration avec une autre personne (différente chaque trimestre) qui
sera votre binôme, et avec qui il faudra se répartir la charge de travail.

\vspace{1em}

Vous l'aurez compris, une des plus grosses difficultés de la 3IF est de réussir
à gérer correctement l'enchaînement des TP de façon à pouvoir tous les
traiter de manière suffisante, alors soyez prévoyants si vous ne voulez pas
exploser votre budget caféine et raccourcir drastiquement votre temps de sommeil ! (Avec
du recul, c'est un peu comme de l'ordonnancement de tâches sur un processeur
 double coeur...) 

\vspace{1em}

Un dernier point : attention à la sur-qualité, c'est de la non-qualité ! Voici un principe de
base en informatique, et on vous le répètera bien assez souvent durant l'année !
