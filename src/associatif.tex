Tu cherches une association particulière ? Alors tu devrais pouvoir la trouver ici !
Plus d'une centaine d'associations se partagent la vedette sur l'INSA, du club photo au club mangas, en
passant par les clubs danse, voile (pour ceux
voulant faire le tour de France à la voile), magie, ou encore ceux
préparant les grands évènements ayant lieu sur le campus, tout y est, il ne
reste plus qu'à réussir à se décider ! Et le choix sera dur...
N'hésites pas, associe-toi à ces valeureux groupes faisant vivre le campus ! (\emph{et accessoirement, gagne un niveau en «~je fais autre chose que
des TP~» !})

Voilà un léger (mais alors infiniment léger) descriptif des assos :

\paragraph{Arts et Spectacles}
Le ciné-club propose des projections de films à prix... minime ?
Des associations musicales et théâtrales n'attendent que toi pour casser les planches et organiser encore plus de merveilleux spectacles.

\paragraph{Culture et Loisirs}
La danse, la lecture, l'astronomie, la musique ou l'électronique t'intéressent ?
Alors ne t'en fais pas, il y a obligatoirement une asso pour toi !
Tu es reporter dans l'âme ? Alors rejoins les rangs de l'Insatiable, le journal gratuit des étudiants, publié penta-annuellement !

\paragraph{Humanitaire et Social}
Bien qu'ingénieur, nous restons humains ! (\emph{ou orangs-outans...})
Le Karnaval Humanitaire organise ainsi chaque année une semaine de festivités,
sous un chapiteau installé sur le campus, afin de financer des projets
humanitaires.
Handizgoud a pour objectif de sensibiliser les futurs cadres à l'insertion
professionnelle des personnes handicapées.

\paragraph{Sports}
Que tu sois débutant ou champion, que tu veuilles faire de l'aviron, du
basket, du rugby, du volley, des combats de robots (\emph{euh non, ça
c'est en IF, et en plus c'est vrai...}), tu pourras trouver tout ce que tu veux !

Pour ceux avides de détails sur les associations, je vous conseille le site 
\url{http://bde.asso.insa-lyon.fr/botinsa/} qui vous fournira toutes les
informations voulues.

