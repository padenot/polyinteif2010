Tu cherches une association ? Alors tu devrais pouvoir la trouver ici !
Plus d'une centaine d'associations se partagent la vedette sur l'Insa, du club
de ... Yo-Yo ! au club photo, en passant par les clubs danse, voile ( pour ceux
voulant faire le tour de France à la voile... ), ou encore les clubs
préparant les grands évènements ayant lieu sur le campus, tout y est, il ne
reste plus qu'à réussir à se décider !
Alors n'hésites pas, associe toi à ses valeureux groupes faisant vivre le campus
par leur valeureuses missions ! ( et gagne un niveau en "je fais autre chose que
des TPs" en passant ! )

Voilà un léger (mais alors infiniment léger) descriptif des assos :

\paragraph{Arts et Spectacles}
Le ciné-club propose des projections de films à prix... minime ?
Des associations musicales,théâtrales n'attendent que toi pour casser les planches et organiser encore plus de merveilleux spectacles.

\paragraph{Culture et Loisirs}
La danse, lecture, astronomie,la musique, l'électronique t'intéresse ?
Alors ne t'en fais pas, il y a obligatoirement une asso pour toi !
L'écriture t'intéresse ? Alors rejoins les rangs de l'Insatiable, le journal des étudiants publiés penta-annuellement !

\paragraph{Humanitaire et Social}
Bien qu'ingénieur, nous restons humains ! (... ou orang-outan.)
Le Karnaval Humanitaire organise ainsi chaque année une semaine de festivités,
sous un chapiteau installé sur le campus, afin de financer des projets
humanitaires au Burkina-Faso.
Handizgoud a pour objectif de sensibiliser les futurs cadres à l'insertion
professionnelle des personnes handicapées.

\paragraph{Sports}
Que tu sois débutant ou champion, que tu veuilles faire de l'aviron, du
basket-ball, du rugby, du volley-ball, des combats de robots (... heu non, ça
c'est en IF, c'est vrai.), tu pourras trouver ce que tu veux par ici !

Pour ceux, avide de détails sur ces associations, je vous envoi sur :
\url{http://bde.asso.insa-lyon.fr/botinsa/} qui vous fournira toutes les
informations voulues.

