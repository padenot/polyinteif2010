\subsection*{Présentation}
\begin{description}
    \item[Nom - Prénom :]
    \item[Surnom]
    \item[Âge :]
    \item[Région d'origine :]
    \item[Sexe :]
    \item[Célibataire~?]
    \item[IF, premier choix ou pas trop ?]
    \item[Des problèmes de santé particuliers ?] (Pour organiser au mieux la
	    semaine).
\end{description}
\subsection*{Photo}
Colle nous donc une jolie photo de toi. Tout contrevenant s'expose à des
poursuites !
\vspace{3cm}
\orga{images/anonymous.jpg}

\subsection*{Test psychologique}
Élaboré par nos plus grands spécialistes, cette section nous permettra d'en
savoir un peu plus sur toi.

\begin{itemize}
    \item \textbf{Donne une description précise du rituel  
	 de l'huitre et du pingouin :}
    \vspace{1cm}

    \item \textbf{Que serais tu si tu étais :}
    \begin{description}
	\item[Chuck Norris:]
	\item[Une série:]
	\item[Un droïd:]
	\item[Un truc vert:]
	\item[Une chanson:]
	\item[Un film:]
	\item[Une heure de la journée:]
    \end{description}
    \item \textbf{Lequel des orgas te parait le plus sympa ?}
    \vspace{1cm}
    \item \textbf{Un ours des forêts du nord met deux jours pour traverser 500m de forêt, pourquoi ?}
    \vspace{2cm}
    \item \textbf{Que cherches tu en venant en IF ?}
    \vspace{3cm}
    \item \textbf{Dessine nous un avion à réaction robotisé intelligent avec une
    lueur maléfique dans les yeux, et des poils soyeux :}
    \vspace{5cm}
    \item \textbf{Pose nous une question :}
    \vspace{2cm}
\end{itemize}
\subsection*{Test de Geekness}
\begin{itemize}
    \item \textbf{Quelle est la différence entre un geek et un nerd ?}
    \vspace{1cm}
    \item \textbf{Combien de langage de programmation maîtrises-tu ?}
    \vspace{1cm}
    \item \textbf{Explique en 3 points pourquoi les produits Apple ne valent
	pas un clou :}
    \vspace{2cm}
    \item \textbf{Combien d'épisode de série regardes-tu en moyenne chaque
	semaine ?} (\emph{parce qu'on est un peu à cours, dans l'équipe})
    \vspace{2cm}
    
    \item \textbf{Tu utilises plutôt quotidiennement :}
    \begin{description}
	\item[$\square$] GNU/Linux.
	\item[$\square$] Microsoft Windows.
	\item[$\square$] Apple MacOS.
	\item[$\square$] BSD -- Solaris -- un Atari -- un Amiga.
	\item[$\square$] Un verre à pintes.
	\item[$\square$] Mais qu'est-ce que je fais là ?!
    \end{description}

    \item \textbf{L'info, pour toi, c'est :}
    \begin{description}
	\item[$\square$] Une passion.
	\item[$\square$] Un moyen de se faire pas mal d'argent.
	\item[$\square$] Un bon moyen de faire du management plus tard.
	\item[$\square$] Ta vie.
	\item[$\square$] Un outil.
	\item[$\square$] Marrant.
	\item[$\square$] 42.
    \end{description}
    \item \textbf{T'as deux heures à tuer, tu fais quoi ?}
    \begin{description}
	\item[$\square$] T'as justement une classe à écrire sur un de tes
	projets.
	\item[$\square$] Tu prends un bon bouquin.
	\item[$\square$] T'appelles un pote et tu vas prendre un café en ville.
	\item[$\square$] Tu joues à un jeu (vidéo, ou pas).
	\item[$\square$] Tu fais du sport.
	\item[$\square$] Tu crées un truc (dessin, musique, cuisine, UML, ...).
    \end{description}

    \item \textbf{Comment s'appelle ceci ?}
    \begin{verbatim}
	#include<stdio.h>
	char*f="char*f=%c%s%c;main()"
	"{printf(f,34,f,34,10);}%c";
	main(){printf(f,34,f,34,10);}
    \end{verbatim}
    \vspace{1cm}
    \item \textbf{Une passion dévorante ?} (sport, musique, dessin, etc.)
    \vspace{2cm}
    \item \textbf{Des projets notables en info ?} (i.e. utilisé par plus de monde que ta
	    grand mère, toi, et le prof qui t'as corrigé...)
    \vspace{2cm}
\end{itemize}
\subsection*{Encore photo}
Colle ici une photo de toi complètement stupide :
\begin{center}
\includegraphics[height=5cm, angle=120]{images/anonymous.jpg}
\end{center}

Les meilleurs photos gagneront un passage sur
\url{http://bonjourlesifs.tumblr.com}.

\subsection*{Presque la fin...}
Marque nous quelque chose. Oui, n'importe quoi. Non, vraiment, tu peux te
lâcher, vas-y. Y'a même de la place pour dessiner, écrire de la musique, cet
espace est à toi !
\vspace{8cm}
\subsection{Mais qu'est-ce que je fais avec toutes ces questions ?}
C'est pas bien compliqué, tu les renvoie avec le coupon réponse pour le WEI et
ton chèque à :
\adresseCoupon



Fais quand même attention, le chèque pour le WEI est à mettre à l'ordre de AEDI, pas de la personne ci-dessus !

\vfill
\columnbreak
~
\vfill
\hspace{-5cm}
\includegraphics[height=8cm]{images/stormTrooperBourre.png}


