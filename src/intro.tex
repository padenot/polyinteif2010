Aïe aïe aïe, ça y est tu as reçu la confirmation que l'année prochaine tu seras à l'INSA de Lyon, en département Informatique en plus...

\vspace{1em}

Laisse moi te dire que tu es sacrément -- suspense -- bien tombé(e) ! Alors
bienvenue dans ta nouvelle vie !


\vspace{1em}

Pour certains c'est l'occasion de changer d'environnement, d'entamer une
nouvelle page de sa vie,  d'autres connaissent déjà un peu la maison mais
ne savent pas vraiment où ils vont en ayant coché la case «~Département
Informatique~» en fin de deuxième année. Vous allez donc vous retrouver dans
une promo composée de gens d'origines variées, qui seront comme vous parachutés
dans cette nouvelle vie qu'est la vie d'étudiant à l'INSA de Lyon !


\vspace{1em}

Bien entendu les questions fusent... Comment ça va se passer l'année
prochaine ? Est-ce que je vais me retrouver tout seul ?  Est-ce que les gens
sont sympa ? Est-ce que  Lyon c'est bien ? Comment organiser mon arrivée du
mieux possible ? Est-ce que les loutres volent mieux au parc de la Feyssine
qu'au parc de la Tête d'or ?


\vspace{1em}

Toutes ces interrogations, aussi capitales soient-elles pour ton arrivée et ta
nouvelle scolarité, vont trouver réponse ici-même dans ce magnifique
«\texttt{polydintegration2010.pdf}» que tu viens d'ouvrir ! 


\vspace{1em}

Ainsi au travers de ces quelques pages, rédigées avec amour par tes futurs
hypothétiques parrains, nous allons te présenter non seulement l'organisation
de ta future (\emph{et géniale !}) nouvelle année, mais aussi toute ta première
semaine, la semaine d'intégration  !


\vspace{1em}

Et là, je vois tout autour de moi les visages pâlir, les gens se mettre à
trembler de peur, le ciel s'assombrit tout autour de vous, un orage se
déclenche, Céline Dion se remet à chanter, vous réalisez d'un coup : ça y est,
je vais être bizuté !


\vspace{1em}

Ce à quoi nous répondons :

\begin{citationi}{ \reflectbox{©}\emph{padenot} }
    Alors là, déjà, NON ! 
\end{citationi}

Après cette superbe citation, que vous aurez l'occasion d'entendre à de
nombreuses reprises, nous allons expliquer un peu ce que c'est que la «~semaine
d'intégration~».

On est tous passé par l'étape de devoir «~changer de vie~», devoir quitter notre
cocon familial, notre petit(e) FAC/IUT/Prépa , notre petite ville qu'on
connaissait si bien et en même temps qu'on connaissait trop.
Et maintenant, paf pastèque, ça arrive d'un coup, on va se retrouver parachuté
à Lyon et à l'INSA en plus, loin de cette «~petite ville qu'on aimait tant~»,
sans vraiment connaître grand monde...

Tout ça pour dire qu'on a tous à un moment ou à un autre été à votre place ...
du moins on se l'est tous imaginés, car ça c'était sans compter cette fameuse
semaine d'intégration que nos très chers parrains nous avaient préparée ! Grâce
à elle nous avons pu créer des liens forts entre nous et devenir la promo
soudée que vous allez rencontrer dès Septembre. Et pour chacun d'entre nous,
cela reste un souvenir d'une période excellente, un moment fort de notre
vie probablement !

Toutes les animations sont faîtes pour que vous puissiez rencontrer les
personnes qui deviendront peut-être d'excellents amis (\emph{nous c'est arrivé,
vous ça arrivera !}), vos futurs camarades de groupe (\emph{nous c'est
arrivé, vous ça arrivera !}), vos futurs binômes (\emph{je suis lourd
hein !}), et peut-être plus...

Alors rassurez vous, pas de bizutage avec nous, uniquement des jeux,
sorties, animations, soirées et un week end d'intégration de folie, qui
vous feront vous sentir chez vous dès les premières minutes de votre vie
insalienne.

Nous avons eu droit à ca l'année dernière, et devinez quoi : Notre objectif
cette année c'est de faire pareil pour vous ! Alors dévorez vite le poly afin
de découvrir à quelle sauce vous allez être mangés en septembre !

Il ne me reste qu'à vous souhaiter d'excellentes vacances et à vous donner
rendez vous dès la troisième semaine de Septembre !

Parce que nous, au nom de toute l'équipe : on vous attend avec impatience bande
de loutres !
\vspace{1cm}
\begin{flushright}
\emph{Juju, Resp'inté 2010}
\end{flushright}
\newpage
