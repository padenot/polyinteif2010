Aïe aïe aïe, ça y est tu as reçu la confirmation que l'année prochaine tu seras à l'INSA de Lyon, au département Informatique en plus...

\vspace{1em}

Laisse-moi te dire que tu es sacrément -- suspense -- bien tombé(e), alors
bienvenue dans ta nouvelle vie !


\vspace{1em}

Pour certains c'est l'occasion de changer d'environnement, d'entamer une
nouvelle page de leur existence, tandis que d'autres connaissent déjà un peu la maison mais
ne savent pas exactement dans quoi ils se sont engagés en ayant coché la case «~Département
Informatique~» en fin de deuxième année de prépa intégrée. Tu vas donc te retrouver dans
une promo composée de gens aux origines variées, qui seront comme toi parachutés
dans cette nouvelle vie d'étudiant IF à l'INSA de Lyon !


\vspace{1em}

Bien entendu les questions fusent... Comment ça va se passer l'année
prochaine ? Est-ce que je vais me retrouver tout seul ?  Est-ce que les gens
sont sympas ? Est-ce que  Lyon c'est bien ? Comment organiser mon arrivée du
mieux possible ? Est-ce que les loutres volent mieux au parc de la Feyssine
qu'au parc de la Tête d'or ?


\vspace{1em}

Toutes ces interrogations, aussi capitales soient-elles pour ton arrivée que pour ta
scolarité, vont trouver réponse ici-même dans ce magnifique
«\texttt{polydintegration2010.pdf}» que tu viens d'ouvrir !


\vspace{1em}

Ainsi, au travers de ces quelques pages rédigées avec amour par tes futurs
hypothétiques parrains, nous allons te présenter non seulement l'organisation
de l'année (\emph{géniale}) à venir, mais aussi (et surtout) le déroulement de sa première
semaine : la semaine d'intégration !


\vspace{1em}

Et là, je vois tout autour de moi les visages pâlir, les gens se mettre à
trembler de peur, le ciel qui s'assombrit, un orage qui se
déclenche, Céline Dion qui se remet à chanter, et tu réalises tout à coup : ça y est,
je vais être bizuté !


\vspace{1em}

Ce à quoi nous répondons :

\begin{citationi}{ \reflectbox{©}\emph{padenot} }
    Alors là, déjà, NON !
\end{citationi}

Après cette superbe citation, que tu auras l'occasion d'entendre à de
nombreuses reprises, nous allons t'expliquer ce que c'est que la «~semaine
d'intégration~».

On est tous passés par l'étape du «~changement de vie~», devoir quitter notre
cocon familial, notre petit(e) FAC/IUT/Prépa , notre chère ville qu'on
connaissait si bien et en même temps qu'on connaissait déjà trop.
Et maintenant, paf, ça arrive d'un coup, on va se retrouver parachuté
à Lyon, à l'INSA qui plus est, loin de cette «~petite ville qu'on aimait tant~»,
sans vraiment connaître grand monde dans le coin...

Tout ça pour dire qu'on a tous à un moment ou à un autre été à ta place.
Du moins on se l'est tous imaginé, car ça c'était sans compter cette fameuse
semaine d'intégration que nos très chers parrains nous avaient préparée ! Grâce
à elle, nous avons pu créer entre nous des liens forts, et devenir la promo
soudée que tu vas rencontrer dès septembre. Pour chacun,
cela reste un excellent souvenir, un des moments forts de notre vie !

Toutes les animations sont faites pour que tu puisses rencontrer les
personnes qui deviendront peut-être d'excellents amis (\emph{ça nous est arrivé,
ça vous arrivera !}), tes futurs camarades de groupe (\emph{ça nous est arrivé,
ça vous arrivera !}), tes futurs binômes (\emph{je suis lourd
hein !}), et peut-être même plus...

Alors rassure-toi, pas de bizutage avec nous, uniquement des jeux,
sorties, animations, soirées, et un week end d'intégration de folie, qui
te feront te sentir chez toi dès les premières minutes de ta vie
Insalienne.

Nous avons eu droit à ça l'année dernière, et devine quoi : notre objectif
cette année c'est de faire pareil pour toi, mais en mieux ! Alors dévore vite le poly afin
de découvrir à quelle sauce tu vas être mangé !

Il ne me reste qu'à te souhaiter d'excellentes vacances et à te donner
rendez-vous dès la deuxième semaine de septembre !

Parce que nous, au nom de toute l'équipe : on vous attend avec impatience, bande
de loutres !
\vspace{1cm}
\begin{flushright}
\emph{Juju, Resp'inté 2010}
\end{flushright}
\newpage
