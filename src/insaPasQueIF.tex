Le département IF, c'est 120 élèves par promo, certes, mais les autres départements dans tout ça ?
Car oui, il y a 11 autres départements à l'INSA, pour un total de 800 étudiants par promo.
Peut-être te dis-tu que "Ce ne sont pas des IFs donc on ne les vera jamais !"
Et bien détrompe-toi, tu pourras les rencontrer dans les résidences, les restos,
les associations, les chouilles (\emph{surtout pendant les chouilles...}), mais
aussi les... cours de sport inter-départ' ou encore les cours de langues
étrangères! (\emph{oui c'est important de préciser étrangères pour éviter les
ambiguïtés...}) Vous serez alors dans une atmosphère conviviale pour
retrouver les joies de votre LV2 favorites.

Outre ces autres départements, il y a aussi les fameux PC ou encore Premier
Cycle qui correspondent à la prépa intégrée INSA et qui consiste à préparer les
étudiants à entrer dans les différents départements que compte l'INSA à travers
toute la France ! Pendant ces deux ans, ils n'ont pas étudié exactement les
mêmes disciplines qu'un élève de prépa en lycée aura pu étudié. Ils font bien
évidemment des maths, de la physique, et toutes ces matières que les BTS/DUT
adorent, mais également du dessin industriel ou encore de la programmation en
PASCAL... (\emph{yeah c'est la fête !}).


Ainsi, ne soyez pas étonnés de voir que certains IFs connaîtront déjà des
personnes de ces départements, car oui, une grande proportion a fait le Premier
Cycle (environ 85 \%). Mais rassurez-vous, ils sont tellement nombreux  (plus de
800 par promo) qu'ils ne se connaissent pas tous et par conséquent, vous
ne serez pas confrontés à un groupe soudé dès le départ. Ainsi, pas de souci
pour s'intégrer parmi eux. Et puis pour tout vous dire, ils sont plutôt
accueillant, parole d'admis direct.

Une dernière petite chose à noter : il y a des filières internationales à
l'INSA, et ces dernières accueillent une grande proportion d'étudiants étrangers
et.... une bonne partie viennent en IF! Idéal pour se familiariser avec d'autres
cultures! Voici les nationalités que vous pourrez rencontrer par exemple:
Chinoise, Roumaine, Italienne, Brésilienne, Mexicaine, Slovaque, Bretonne,
Tunisienne... Il y a aussi quelques français parmi tout ça.
