Le département IF, c'est environ 120 élèves par promo, certes, mais et les autres départements dans tout ça ?
Car oui, il y a 11 autres départs' à l'INSA, pour un total de 800 étudiants par promo.
Peut-être te dis-tu que «~Ce ne sont pas des IFs donc on ne les verra jamais !~»
Et bien détrompe-toi, tu pourras les rencontrer dans les résidences, les restos,
les associations, les chouilles (\emph{surtout les chouilles...}), mais
aussi les... Cours de sport inter-départ' ou encore les cours de langues
étrangères (\emph{oui, c'est important de préciser étrangères, pour éviter les
ambiguïtés...}) ! Tu seras alors dans une atmosphère conviviale pour
retrouver les joies de ta LV2 favorite.

Outre ces autres départements, il y a aussi les fameux PC (Premier
Cycle) qui sont en prépa intégrée à l'INSA (elle consiste à préparer les
étudiants à entrer dans les différents départements que compte l'INSA à travers
toute la France) ! Pendant ces deux ans, ils n'ont pas étudié exactement les
mêmes disciplines qu'un élève de CPGE aurait pu étudier. Ils font bien
évidemment des maths, de la physique, et toutes ces matières que les BTS/DUT
adorent, mais également du dessin industriel ou encore de la programmation en
PASCAL... (\emph{yeah, c'est la fête !})


Ainsi, ne sois pas étonné de voir que certains IFs connaîtront déjà des
personnes de ces départements, car oui, une grande proportion de la promo a fait le Premier
Cycle (environ 80 \%). Mais rassure-toi, ils sont tellement nombreux  (plus de
800 par promo) qu'ils ne se connaissent pas tous et par conséquent, tu
ne seras pas confrontés à un groupe soudé dès le départ. Ainsi, pas de souci
pour t'intégrer parmi eux. Et puis pour tout te dire, ils sont plutôt
accueillants, parole d'admis direct.

Une dernière petite chose à noter : il y a des filières internationales à
l'INSA, et ces dernières accueillent une grande proportion d'étudiants étrangers, dont
une bonne partie vient en IF ! Idéal pour se familiariser avec d'autres
cultures ! Voici un petit extrait des nationalités que tu pourras rencontrer :
Chinoise, Roumaine, Italienne, Brésilienne, Mexicaine, Slovaque, Bretonne,
Tunisienne... Et il y a aussi quelques français parmi tout ça !
