Être étudiant à l'INSA, c'est également être étudiant à Lyon et de ce fait, ça
catapulte du pâté issu de l'agriculture biologique (c'est bien quoi) !

\vspace{1em}

Bienvenue dans la troisième plus grosse ville de France, une ville
qui bouge énormément, surtout quand comme beaucoup (par exemple l'auteur de ce
paragraphe) tu es issu(e) d'une petite bourgade de 50 habitants et 150 moutons.

\vspace{1em}

Lyon c'est 450000 habitants, 9 arrondissements placés un peu aléatoirement sur
la carte, 2 fleuves, 4 lignes de métro, 4 lignes de tram, une centaine de lignes
de bus régulières et des évènements tout au long de l'année ! Mais c'est aussi 3
universités nommées respectivement Lyon 1 - Université Claude Bernard (c'est
nous ça !), Lyon 2 - Université Lumière et Lyon 3 - Université Jean
Moulin, totalisant près de 125000 étudiants !

\vspace{1em}

Lyon est considérée comme l'embouchure de la vallée du Rhône, ce qui lui confère
une situation géographique extrêmement intéressante ! En TGV, tu es
à 2h de Paris et de Marseille ! Mais tu es également à 1h30 des pistes de ski,
de quoi passer des week-end de détente parfaits entre amis, ou bien choisir de
pratiquer le ski comme activité physique en sport ! (attention, les places sont limitées, alors soyez réactifs)

\vspace{1em}

Ce qui est dingue, c'est qu'avec tout ça, cette ville a l'avantage de conserver
des loyers relativement peu élevés comparés à d'autres grandes villes ! Un bon point pour nous, les étudiants.

\vspace{1em}

Tout d'abord, il te faudra quelque chose d'essentiel pour apprendre à découvrir
Lyon. Ce quelque chose s'intitule «~Le Petit Paumé~», un bouquin façon guide
Michelin qui te renseignera sur l'ensemble des adresses incontournables de ta
nouvelle ville ! Ce guide est gratuit, et est distribué en octobre ou novembre
selon les années dans les endroits stratégiques de Lyon. Notons les
principaux : sur le campus, devant le restaurant Castor \& Polux (le \emph{Beurk} pour les
intimes), au BdE, ou encore Place Bellecour. Profitez-en, c'est gratuit
et c'est rédigé par des étudiants comme vous !

\vspace{1em}

Justement, rentrons un peu plus dans les détails et voyons ce qu'il est
intéressant de savoir à Lyon. Tout d'abord, la ville est construite à la
confluence de deux fleuves, le Rhône et la Saône, donnant naissance à ce que
l'on appelle la «~presqu'île~», correspondant aux premier et deuxième arrondissements.
C'est sur cette presqu'île qu'est concentrée l'essentiel de l'activité festive
de la ville. Envie d'aller boire un verre ? Tu auras le choix parmi des
centaines de bars ! Un petit creux ? Des restaurants à profusions et notamment
les fameux bouchons Lyonnais afin de manger généreusement des spécialités de
Rhône-Alpes (et pas que) pour 10€ ! C'est le cœur de la ville !

\vspace{1em}

Lyon c'est également des coins insolites à découvrir, et une ville extrêmement
touristique. Comment citer Lyon sans parler de la basilique de Fourvière,
surplombant la ville au sommet de la colline de... Fourvière ? Comment citer
Lyon en oubliant de parler de la fameuse Place Bellecour et de la statue de
Louis XIV ? Comment citer Lyon sans parler de la fontaine de Bartholdi (artiste
ayant réalisé la statue de la liberté pour les incultes) située
Place des Terreaux ?

\vspace{1em}

Lyon est une ville très riche historiquement parlant ! Vous pourrez découvrir le
vieux Lyon et ses bâtisses moyenâgeuses en mangeant une glace artisanale. Vous
aurez l'occasion de parcourir les nombreuses traboules, petits chemins étroits, passant dans
des lieux étranges de la ville. Vous
pourrez également contempler de magnifiques trompe l'œil à la Croix Rousse, une
autre colline surplombant la ville, avec notamment la fameuse maison des Canus.

\vspace{1em}

Vous pourrez redécouvrir la ville le soir en vous
baladant sur les bords de la Saône ou lors de la fête des Lumières début décembre,
lorsqu'elle revêt son plus bel apparat pour charmer les touristes
venus du monde entier pour admirer cette mise en scène spectaculaire !
Vous aurez l'opportunité de participer à des centaines de festivals culturels gratuits
! Vous pourrez sortir en boîte sur les péniches, vous remplir les
oreilles de Jazz jusqu'à 5h du matin ou bien faire des barbecues
jusqu'à pas d'heure au parc de Parilly. Vous pourrez essayer de
charmer de jeunes demoiselles, le soir, sur la place Bellevue sur les
pentes de la Croix Rousse. Vous pourrez visiter des vestiges gallo-romains
à Fourvière. Vous pourrez même voir le plus célèbre octogénaire
Lyonnais, Jean-Pierre, courir sur les berges du Rhône tous les soirs et
l'encourager comme le font les centaines d'étudiants venant passer leur
soirée dans l'herbe ! (pour en savoir plus sur notre ami Jean-Pierre : \url{http://bit.ly/cMbVnA})

\vspace{1em}

Et lorsque vous aurez la chance de connaître les grèves des TCL (exploitant du
réseau de transports en commun) et que vous marcherez la nuit, vous vous
rendrez compte à quel point vous avez de la chance d'être venus à Lyon !

