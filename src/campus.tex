Le campus ? Rien de moins que 100 hectares... Et tu n'en verras probablement pas
la moitié !

Le campus est une véritable ville à l'intérieur de Villeurbanne, de la taille
d'un arrondissement Lyonnais en superficie et d'une petite ville
démographiquement! Tu pourras y trouver tout ce dont tu as besoins pour vivre!
Un stock de jeunes filles limité ou un stock de jeunes hommes illimité mais
aussi des laveries se trouvant dans les résidences pour toujours être au top de
la classe avec ton t-shirt du panda roux, un bar associatif du nom de "K-fêt"
pour se jeter un verre vraiment pas cher après un TP prise de tête en écoutant
la bonne vieille fanfare, la Coop (l'épicerie du Bde) pour des petites courses
(\emph{des coursinettes ?}) à un tarif assez avantageux, et bien d'autres choses encore
!

J'allais oublier ! Depuis ta chambre, tu sors la tête de la fenêtre et même pas
besoins de téléphone, tu peux dialoguer en live avec des centaines de nanas et
de gars près de chez toi ! (\emph{enfin surtout des gars}) Finis les SMS surtaxés et
les notes salées.

Alors pourquoi sortir du campus te demanderas-tu ? Mais pourquoi donc se risquer
à la "grande" ville alors que tu n'es qu'un pauvre geek apeuré et effrayé par la
société contemporaine ? Parce que c'est Lyon ? Que c'est une superbe ville ?
Parce que la fac de lettres, accessoirement le repère de toutes les nanas, se
situe au coeur de Lyon ? Parce que pour shaker son booty comme ils disent dans
les clips de rap US il faut bouger sur la presqu'île ? Pour avoir la réponse à
ta question, il te faudra être patient et attendre la partie dédiée à la ville
de Lyon rédigée par un non-lyonnais qui vient de la cambrousse Yconnaise mais
qui connaît quand même Lyon un peu pour en parler!

Oui d'ailleurs petite précision, dans ta promo à l'INSA de Lyon, tu ne trouveras
que peu de Lyonnais... 
