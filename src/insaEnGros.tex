\vspace{1em}
{
    \footnotesize
    \parindent 0em
    \begin{changemargin}{-1cm}{0cm}
\textbf{M}aintenant que tu as accepté, tu te dis peut-être :\\
\textbf{A}lors, l'INSA c'est quoi au juste ? Ou encore...\\
\textbf{Y} aura-t-il ce que je cherche ?\\

\textbf{T}rès bien, voici le moment où tout ceci te sera dévoilé ! (ou pas)\\
\textbf{H}istoriquement, l'INSA a été fondé en 1957... Comment ? Inintéressant ?\\
\textbf{E}t alors ? Un peu d'histoire ça ne fait pas de mal !\\

\textbf{F}ort heureusement (pour toi), on peut raccourcir un peu tout ça.\\
\textbf{O}sons donc passer sous silence l'historique Insalien...\\
\textbf{R}elativement vite, tu t'habitueras au campus de la DOUA\\
\textbf{C}'est là que tu vivras pour la majorité, entouré d'autres étudiant(e)s\\
\textbf{E}videmment me diras-tu...\\

\textbf{B}on, on va peut-être pouvoir commencer, tu ne crois pas ?\\
\textbf{E}t bien ?\\

\textbf{I}nitialement, ça n'était pas une introduction, mais bon...\\
\textbf{F}inalement, ça l'est.\\
\end{changemargin}
} % End parindent

Bon allez, reprenons.

\vspace{1em}

L'Institut National des Sciences Appliquées est né en 1957, de par la volonté de
former des ingénieurs en grand nombre. L'établissement lyonnais fut bientôt 
rejoint par 4 autres, situés à Toulouse, Rennes, Rouen
et Strasbourg (et oui, il n'y en a pas à Paris !).

\vspace{1em}

L'INSA de Lyon, situé à Villeurbanne (\emph{cherchez l'erreur}) admet en son sein plus
de 5000 étudiants et doctorants, et forme des ingénieurs dans 12 domaines de
spécialisation, couvrant toutes les sciences de l'ingénieur (mécanique,
énergétique, chimie, etc.).

\vspace{1em}

Les nombreuses associations Insaliennes font du campus un lieu à la vie
étudiante dynamique et passionnante. Les divers départements constituant
ce fabuleux ensemble se situent sur le domaine scientifique de la DOUA (100 
hectares, et rien de moins que 26000 personnes sur le site).
