Deux choix pour arriver sur lyon :
Débutons par celui posant le plus de problème (encore que) : la voiture !
Alors une fois sur le périphérique lyonnais (oui jusqu'ici, vous devez choisir de tourner à droite ou à gauche en jouant à pile ou face !)
Il vous faut sortir à la «~\textbf{Porte de la Doua}~» ou «~\textbf{Porte de
    Croix-Luizet}~» (peut-être plus facile d'accès...)
Et ensuite, vous suivez les panneaux «~Campus scientifique de la DOUA~»
Vous tombez alors sur l'INSA, et une de ses multiples entrées !

L'autre choix, plus écologique : le train !
Si tu arrives à la gare Perrache, deux solutions :
\begin{itemize}
    \item Soit tu prends le \textbf{métro A} (en rouge sur les plans) afin
    d'aller jusqu'à \textbf{Charpennes - Charles Hernu} puis, tu
    sors du métro et prends le \textbf{Tramway T1} direction
    \textbf{IUT Feyssine}. Tu t'arrêteras alors à l'arrêt
    \textbf{Doua - Gaston Berger}, tu seras alors arrivé(e) sur le campus, au pied de notre cher bâtiment.

    \item Soit tu prends le \textbf{Tramway T1} directement, et tu
    attends l'arrêt La \textbf{Doua - Gaston Berger}. Oui, c'est
    beaucoup plus simple, mais c'est aussi plus long, en fonction des
    temps d'attente aux correspondances.
\end{itemize}


Si tu arrives à la gare Part-Dieu, c'est encore plus simple ! Le
\textbf{Tramway T1} s'arrête pour te prendre juste à la sortie de la gare,
coté \textbf{Vivier Merle}, pour t'emmener de suite à l'INSA (comme
précédemment, arrêt \textbf{La Doua - Gaston Berger}).


Que ce soit en voiture ou en train, nous te conseillons d'acheter un carnet de
10 tickets TCL, ces derniers, au modeste prix de 11.90€ pour les étudiants (pour
les trouver aux bornes, il faut chercher quelques secondes !) te
permettront d'utiliser toutes les correspondances des transports TCL (métro,
tram, funiculaire, bus, trolleybus) et ce pour une durée d'une heure.
(malheureusement, comme vous le verrez rapidement, un retour ne peut se faire
avec le même ticket que l'aller... Alors il faut filouter!).

