Deux choix pour arriver sur Lyon :
Débutons par celui posant le plus de problèmes (encore que) : la voiture !
Une fois sur le périphérique lyonnais (oui jusqu'ici, vous devez choisir de tourner à droite ou à gauche en jouant à pile ou face !)
il vous faut sortir à la «~\textbf{Porte de la Doua}~» ou «~\textbf{Porte de
    Croix-Luizet}~» (peut-être plus facile d'accès...)
et ensuite, en suivant les panneaux «~Campus scientifique de la DOUA~»
vous tomberez sur l'une des multiples entrées de l'INSA !

L'autre choix, plus écologique : le train !
Si tu arrives à la gare Perrache, deux solutions :
\begin{itemize}
    \item Soit tu prends le \textbf{Métro A} (en rouge sur les plans) afin
    d'aller jusqu'à \textbf{Charpennes - Charles Hernu} pour une
    petite correspondance avec le \textbf{Tramway T1} direction
    \textbf{IUT Feyssine}. Tu sortiras alors à l'arrêt
    \textbf{La Doua - Gaston Berger}, et tu seras arrivé(e) directement sur le campus, au pied de notre cher bâtiment.

    \item Soit tu prends directement le \textbf{Tramway T1}, et tu
    attends l'arrêt \textbf{La Doua - Gaston Berger}. Oui, c'est
    beaucoup plus simple, mais c'est aussi plus long en fonction des
    temps d'attente aux correspondances.
\end{itemize}


Si tu arrives à la gare Part-Dieu, c'est encore plus simple ! Le
\textbf{Tramway T1} s'arrête pour te prendre juste à la sortie de la gare,
coté \textbf{Vivier Merle}, pile devant le centre commercial \textbf{La Part-Dieu}, pour
t'emmener tout de suite à l'INSA (comme précédemment, arrêt \textbf{La Doua - Gaston Berger}).


Que ce soit en voiture ou en train, nous te conseillons d'acheter un carnet de
10 tickets TCL, au modeste prix de 11.90€ pour les étudiants (pour
les trouver aux bornes, il faut chercher quelques secondes !). Ces derniers te
permettront d'utiliser toutes les correspondances des transports TCL (métro,
tram, funiculaire, bus, trolleybus) et ce pour une durée d'une heure par ticket 
(malheureusement, comme tu le verras rapidement, un retour ne peut se faire
avec le même ticket que l'aller... Alors il faut filouter !).

