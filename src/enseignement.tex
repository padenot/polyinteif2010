En IF, on fait de l'informatique, certes, mais c'est quand même un domaine assez
large. Les matières étudiées en 3IF sont (en gros) :
\begin{itemize}
    \item La programmation (C, C++, Java).
    \item La conception (UML, Merise) et le génie logiciel.
    \item La gestion des données (SQL, XML, administration, algèbre relationnel).
    \item L'algorithmie (mathématiques discrètes, intelligence artificielle, imagerie numérique).
    \item L'architecture des ordinateurs.
    \item Les réseaux, les systèmes d'exploitation.
\end{itemize}

L'accent est mis sur l'acquisition de méthodes et sur la généralité, pour savoir faire beaucoup
de choses différentes. Bien qu'il soit utile de connaître le \texttt{C++} ou le \texttt{SQL} sur le bout
des doigts, la notation ne se fait que rarement sur le code, et même si on pourrait croire au premier abord les
IUT ont l'avantage, il n'en est rien au moment des résultats.

Tout cela est complété par une bonne dose de cours de maths supplémentaires : probabilités,
traitement du signal, analyse numérique, qui trouveront toujours une application dans les TP.

Concernant les DS, le début de l'année est plutôt calme, et les premiers
commencent assez tard. Certaines matières voient leurs évaluations se
passer très tard dans l'année alors que le cours a été fait beaucoup plus tôt,
d'où l'intérêt de prendre efficacement des notes lors des amphis/TD, afin de ne
pas perdre son temps au moment des révisions.

La charge de travail est assez irrégulière, et il est capital de bien s'organiser pour
pouvoir réussir au mieux et ne pas passer ses nuits à finir un TP. Un bon conseil : ne remets
jamais au lendemain ce que tu peux faire le jour même !
