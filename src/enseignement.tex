En IF, on fait de l'informatique, certe, mais c'est quand même un domaine assez
large. Les domaines étudiés en 3IF sont :
\begin{itemize}
    \item La programmation (à travers C++, Java, C).
    \item La conception (UML, MERISE) et le génie logiciel.
    \item Les gestion de données (SQL, XML, administration, algèbre relationnel).
    \item L'algorithmie (Maths discrètes, intelligence artificielle, imagerie).
    \item L'architecture des ordinateurs.
    \item Les réseaux, les systèmes d'exploitation.
\end{itemize}

L'accent est mis sur l'acquisition de méthodes, et la généralité. Bien qu'il
soit utile de connaitre \texttt{C++} ou \texttt{SQL} sur le bout des doigts, la notation ne se
fait que rarement sur le code, et même si on pourrait croire les DUT pourraient
avoir l'avantage, il n'en est rien au moment des résultats.

Tout cela est completé par une bonne dose de cours de maths supplémentaire : probabilités,
traitement du signal, qui trouverons toujours une applications dans des TP.

Concernant les DS, le début de l'année est plutôt calme, et les premiers
commencent assez tard dans l'année. Certaines matières voient leurs évaluation se
passer très tard dans l'année alors que le cours a été fait beaucoup plus tôt,
d'où l'intérêt de prendre efficacement des notes lors des TD, afin de ne
pas perdre son temps au moment des révisions.

La charge de travail est irrégulière, et il est capital de bien s'organiser pour
pouvoir réussir au mieux, et ne pas passer ses nuits à finir un TP.
